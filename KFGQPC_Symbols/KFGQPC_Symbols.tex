% !TeX program = xelatex
% http://fonts.qurancomplex.gov.sa/?page_id=486
\documentclass[oneside]{article}
\setlength{\parindent}{0in}
\usepackage[a4paper, bottom=3.5cm, top=3.5cm]{geometry}
\usepackage{changepage}

%****************************************
%      Packages
%******************************
\usepackage{array}
\usepackage{longtable, booktabs}
\usepackage{libertine}
\usepackage{polyglossia}
\usepackage{import}
\usepackage{url}
\usepackage{fontspec}
\usepackage{listings,color}

% Code environment
\definecolor{verbgray}{gray}{0.9}
\lstnewenvironment{code}{%
\lstset{backgroundcolor=\color{verbgray},
  frame=single,
  framerule=0pt,
  basicstyle=\ttfamily,
  columns=fullflexible}}{}
\definecolor{shadecolor}{rgb}{.9, .9, .9}
% End

\usepackage[%
  pdfauthor={Muḥammad Khālid Ḥussain},
  pdftitle={KFGQPC Arabic Symbols 01 Glyph Table},
  pdfsubject={Font Reference},
  pdfkeywords={font, KFGQPC, symbols, arabic},
  pdfstartview=FitH,
  pdfdisplaydoctitle=true,
  colorlinks=true,
  urlcolor=blue
]{hyperref}

%****************************************
%      Fonts
%******************************
% Polyglossia
\setmainlanguage{english}
\setotherlanguage[calendar=hijri, numerals=mashriq]{arabic}

% Monospaced font
\setmonofont{Inconsolata}

% The KFGPC Arabic Symbols font
\newfontfamily\QPCSymbols[Scale=2.2]{KFGQPC Arabic Symbols 01}
\newfontfamily\QPCSymbolsBig[Scale=3]{KFGQPC Arabic Symbols 01}

% Normal Arabic font
\newfontfamily\arabicfont[%
  Script=Arabic,%
  Numbers=Proportional,%
  Scale=1.6%
]{Arabic Typesetting} % Use if font is available
%]{Scheherazade} % Otherwise use this

%****************************************
%      Metadata
%******************************
\author{Muḥammad Khālid Ḥussain}
\title{KFGQPC Arabic Symbols 01 Glyph Table}
\date{\Hijritoday}%[1]}

\begin{document}
\maketitle

\begin{center}
\QPCSymbolsBig{\XeTeXglyph 3}
\end{center}

\section{Introduction}

This project aims to make the glyphs from the \verb$KFGQPC Arabic Symbols 01$ 
font more accessible to users in \XeLaTeX, while at the same time providing a 
higher quality reference for those using Microsoft Word.\\

As of writing, the font can be downloaded from\\
\url{http://fonts.qurancomplex.gov.sa/download/}\\

The source files for this document can be found at\\
\url{https://github.com/khalid-hussain/Latex-Utilities/tree/master/KFGQPC_Symbols}

\section{Using with \XeLaTeX}

To use this font with XeLaTeX, define the font first. An example of such a 
definition is as follows.

  \begin{code}
  \newfontfamily\QPCSymbols[
    Scale=2.2,
  ]{KFGQPC Arabic Symbols 01}
  \end{code}

Then simply call \verb$\XeTeXglyph <number>$ to call the appropriate glyph into 
your document. If you wish to use the glyps in between other text, it is 
recommended to create a macro. An example of a macro which builds upon the last 
example is as follows.

  \begin{code}
  \newcommand{\<your_command_name>}[1]{{\QPCSymbols{\XeTeXglyph <number>}}}
  \end{code}

Do not forget to scale the glyph according to main text font, otherwise, you 
will have gaps between sentences to make up for the difference in 
heights.

\section{Using with Microsoft Word}

To use the glyphs in Microsoft Word, type the corresponding key and change its 
font to the \verb$KFGQPC Arabic Symbols 01$ font.\\

Credits\\
\url{http://fonts.qurancomplex.gov.sa/?page_id=486}

\newpage

\section{Glyph Table}

\input{table}

\end{document}